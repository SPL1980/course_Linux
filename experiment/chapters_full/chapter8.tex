\chapter{shell脚本编程}

%\vspace{0.2in}
\noindent
一、实验要求
\begin{enumerate}
  \item 掌握shell脚本的结构和运行方法。
  \item 掌握shell函数的使用方法。
  \item 掌握shell脚本中的流程控制。
  \item 掌握shell脚本的调试方法。
\end{enumerate}

\vspace{0.2in}
\noindent
二、实验准备
\begin{enumerate}
  \item 安装有CentOS 7的计算机。
\end{enumerate}

\vspace{0.2in}
\noindent
三、实验内容

\vspace{0.1in}
(一)创建一个简单的shell脚本
\begin{enumerate}
  \item 在主目录中创建bin目录,并进入其中。
  \item 创建一个空白文档,向其中输入下面几行内容:
\begin{lstlisting}[language=bash]
#!/bin/bash

# ~/bin/dirinfo.sh
# 20150501
# A really stupid script to give some basic info about the current directory

pwd
ls
\end{lstlisting}
  \item 把文件保存为dirinfo.sh
  \item 利用下面的命令使文件可执行:\verb|chmod a+x dirinfo.sh|。
  \item 运行该脚本:\verb|./dirinfo.sh|。
\end{enumerate}

\vspace{0.1in}
(二)传递参数给函数
\begin{enumerate}
  \item 在主目录的bin目录中,创建一个空白文档,输入以下代码,并将其保存为func.sh。
\begin{lstlisting}[language=bash]
#!/bin/bash

# func

# A simple function

repeat ( ) {
  echo -n "I don't know $1 $2"
}

repeat Your Name
\end{lstlisting}
  \item 使脚本变成可执行文件:\verb|chmod 755 ~/bin/func.sh|。
  \item 在命令行运行脚本:\verb|~/bin/func.sh|。
\end{enumerate}

\vspace{0.1in}
(三)使用嵌套函数
\begin{enumerate}
  \item 在主目录的bin目录中,创建一个空白文档,输入以下代码,并将其保存为nested.sh。
\begin{lstlisting}[language=bash]
#!/bin/bash

# nested
# Calling one function from another

number_one ( ) {
  echo "This is the first function speaking..."
  number_two
}

number_two ( ) {
  echo "This is now the second function speaking..."
}

number_one
\end{lstlisting}
  \item 使脚本变成可执行文件。
  \item 在命令行中运行该脚本。
\end{enumerate}

\vspace{0.1in}
(四)处理作用域
\begin{enumerate}
  \item 在主目录的bin目录中,创建一个空白文档,输入以下代码,并将其保存为scope.sh。
\begin{lstlisting}[language=bash]
#!/bin/bash

# scope
# dealing with local and global variables

scope ( ) {
  local lclVariable=1
  gblVariable=2
  echo "lclVariable in function = $lclVariable"
  echo "gblVariable in function = $gblVariable"
}

scope

# We now test the two variables outside the function block to see what happens

echo "lclVariable outside function = $lclVariable"
echo "gblVariable outside function = $gblVariable"

exit 0
\end{lstlisting}
  \item 使脚本变成可执行文件。
  \item 在命令行中运行该脚本。
\end{enumerate}

\vspace{0.1in}
(五)使用getopts
\begin{enumerate}
  \item 在主目录的bin目录中,创建一个空白文档,输入以下代码,并将其保存为get.sh。
\begin{lstlisting}[language=bash]
#!/bin/bash

# get
# A script for demonstrating getopts

while getopts "xy:z:" name
do
  echo "$name" $OPTIND $OPTARG
done
\end{lstlisting}
  \item 使脚本变成可执行文件。
  \item 在命令行调用脚本,传递一些参数,如:\verb|~/bin/get.sh -xy "one" -z "two"|。
\end{enumerate}

\vspace{0.1in}
(六)用trap命令清除临时文件
\begin{enumerate}
  \item 在主目录的bin目录中,创建一个空白文档,输入以下代码,并将其保存为sigtrap.sh。
\begin{lstlisting}[language=bash]
#!/bin/bash

# sigtrap
# A small script to demonstrate signal trapping

tmpFile=/tmp/sigtrap$$
cat > $tmpFile

function removeTemp( ) {
  if [ -f "$tmpFile" ]
  then
    echo "Sorting out the temp file..."
    rm -f "$tmpFile"
  fi
}

trap removeTemp 1 2

exit 0
\end{lstlisting}
  \item 使脚本变成可执行文件。
  \item 在命令行中运行该脚本。
  \item 在命令行输入一些文本然后按下【Ctrl+D】组合键。一旦按下这个组合键,就会立即检查在tmp目录中创建的文件(它的名字可能和sigtrap(procid)差不多)。
  \item 现在再次运行代码,但在按下【Ctrl+D】组合键之前,先按下【Ctrl+C】组合键。这一次会收到一条消息``Sorting out the temp file...''。如果此时去查看名称类似于sigtrap(procid)的文件,将发现它并不存在。
\end{enumerate}

\vspace{0.1in}
(七)在数组中使用文本文件中的数据
\begin{enumerate}
  \item 在主目录的tmp目录(如果不存在,就先创建该目录)中,创建一个空白文档,输入以下文本,并将其保存为sports.txt。
\begin{verbatim}
rugby hockey swimming
polo cricket squash
basketball baseball football
\end{verbatim}
  \item 在主目录的bin目录中,创建一个空白文档,输入以下代码,并将其保存为array.sh。
\begin{lstlisting}
#!/bin/bash

# array
# A script for polulating an array from a file

populated=($(cat ~/tmp/sports.txt | tr '\n' ' '))
echo ${populated[@]}

exit 0
\end{lstlisting}
  \item 使脚本变成可执行文件。
  \item 在命令行中运行该脚本。
\end{enumerate}

\vspace{0.1in}
(八)shell脚本实例
\begin{lstlisting}[language=bash]
# 模拟Linux登录shell
#!/bin/bash
echo -n "login:"
read name
echo -n "password:"
read passwd
if [ $name = "cht" -a $passwd = "abc" ]
then
  echo "the host and password is right!"
else
  echo "input is error!"
fi

# 比较两个数的大小
#!/bin/bash
echo "please enter two number"
read a
read b
if (test $a -eq $b)
  then echo "NO.1 = NO.2"
elif (test $a -gt $b)
  then echo "NO.1 > NO.2"
else echo "NO.1 < NO.2"
fi

# 查找/root/目录下是否存在该文件
#!/bin/bash
echo "enter a file name:"
read a
if (test  -e /root/$a)
  then echo "the file is exist!"
else echo "the file is not exist!"
fi

# for循环的使用
#!/bin/bash
clear
for num in 1 2 3 4 5 6 7 8 9 10
do
  echo "$num"
done

# 查看是否是当前用户
#!/bin/bash
echo "Please enter a user:"
read a
b=$(whoami)
if (test $a = $b)
  then echo "the user is running."
else echo "the user is not running."
fi

# 删除当前目录下大小为0的文件
#!/bin/bash
for filename in `ls`
do
  if (test -d $filename)
    then b=0
  else
    a=$(ls -l $filename | awk '{ print $5 }')
    if (test $a -eq 0)
      then rm $filename
    fi
  fi
done

# 普通无参数函数
#!/bin/bash
p ( ) {
  echo "hello"
}
p

# 给函数传递参数
#!/bin/bash
p_num ( ) {
  num=$1
  echo $num
}
for n in $@
do
  p_num $n
done

# 创建文件夹
#!/bin/bash
while :
do
  echo "please input file's name:"
  read a
  if (test -e /root/$a)
  then
    echo "The file is existing! Please input new file name!"
  else
    mkdir $a
    echo "mkdir sussesful!"
    break
  fi
done

# 获取本机IP地址
#!/bin/bash
ifconfig | grep "inet addr:" | awk '{ print $2  }'| sed 's/addr://g'

# 查找最大文件
#!/bin/bash
a=0
for name in *.*
do
  b=$(ls -l $name | awk '{print $5}')
  if (test $b -gt $a)
  then a=$b
    namemax=$name
  fi
done
echo "the max file is $namemax"

# case语句练习
#!/bin/bash
clear
echo "enter a number from 1 to 5:"
read num
case $num in
  1) echo "you enter 1"
  ;;
  2) echo "you enter 2"
  ;;
  3) echo "you enter 3"
  ;;
  4) echo "you enter 4"
  ;;
  5) echo "you enter 5"
  ;;
  *) echo "error"
  ;;
esac

# yes/no返回不同的结构
#!/bin/bash
clear
echo "enter [y/n]:"
read a
case $a in
  y|Y|Yes|YES) echo "you enter $a"
  ;;
  n|N|NO|no) echo "you enter $a"
  ;;
  *) echo "error"
  ;;
esac

# 内置命令的使用
#!/bin/bash
clear
echo "Hello, $USER"
echo "Today's date is $(date)"
echo "the user is :"
who
echo "this is $(uname -s)"
echo "that's all folks!"

# 输入分数显示好坏
#!/bin/bash
echo "Please enter score: "
read score
if [ $score -lt 80  ]
then
  echo "bad!!"
elif [ $score -ge 80 -a $score -lt 90  ]
then
  echo "good!!"
else
  echo "very good!!"
fi

# 编辑服务列
#!/bin/bash
echo "Services: "
echo -n "1) ls "
echo -n "2) ls -l"
echo -n "3) exit"
echo "Please choice [1-3]"
read choice
case $choice in
  1) ls
  ;;
  2) ls -l
  ;;
  3)exit
  ;;
  *) echo "wrong choice"
  ;;
esac

# Fork炸弹:总共只用了13个字符(包括空格)
# 警告:切勿尝试,后果自负!
:(){ :|:& };:
# 注解如下
:()      # 定义函数,函数名为":",即每当输入":"时就会自动调用{}内的代码
{        # ":"函数起始字符
  :      # 用递归方式调用":"函数本身
  |      # 并用管道(pipe)将其输出引至...(因为有一个管道操作字符,因此会生成一个新的进程)
  :      # 另一次递归调用的":"函数
# 综上,":|:"表示的即是每次调用函数":"的时候就会产生两份拷贝
  &      # 调用后台运行字符,以使最初的":"函数被关闭后为其所调用的两个":"函数还能继续执行
}        # ":"函数终止字符
;        # ":"函数定义结束后将要进行的操作...
:        # 调用":"函数,"引爆"fork炸弹
\end{lstlisting}

\vspace{0.1in}
(九)shell脚本的调试

练习理论课中演示的shell脚本,并对所有脚本进行调试。

