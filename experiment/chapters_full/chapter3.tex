\chapter{Linux命令行界面的基本操作}

{\itshape
图形化用户界面下用户操作非常简单而直观,但到目前为止图形化用户界面还不能完成所有的操作任务。字符界面占用资源少,启动迅速,对于有经验的管理员而言,字符界面下使用shell命令更为直接高效。

shell命令是Linux操作系统的灵魂,灵活运用shell命令可完成操作系统所有的工作。并且,类Unix的操作系统在shell命令方面具有高度的相似性。熟练掌握shell命令,不仅有助于掌握CentOS 7,而且几乎有助于掌握各种发行版本的Linux,甚至Unix。

包括CentOS 7在内的Linux系统都具有虚拟终端。虚拟终端为用户提供多个互不干扰、独立工作的工作界面,并且在不同的工作界面可用不同的用户身份登录。也就是说,虽然用户只面对一个显示器,但可以切换到多个虚拟终端,好像在使用多个显示器。

CentOS 7中不仅可在字符界面下使用shell命令,还可借助于桌面环境下的终端工具使用shell命令。桌面环境下的终端工具中使用shell命令可显示中文,而字符界面下默认仅显示英文。
}

\vspace{0.2in}
\noindent
一、实验要求
\begin{enumerate}
  \item 掌握图形化用户界面和字符界面下使用shell命令的方法。
  \item 掌握ls、cd等shell命令的功能。
  \item 掌握软链接和硬链接的使用与区别。
  \item 掌握shell命令挂载和卸载移动存储介质的方法。
\end{enumerate}

\vspace{0.2in}
\noindent
二、实验准备
\begin{enumerate}
  \item 安装有CentOS 7的计算机。
\end{enumerate}

\vspace{0.2in}
\noindent
三、实验内容

\vspace{0.1in}
(一)图形化用户界面下的shell命令操作
\begin{enumerate}
  \item 显示系统时间,并将系统时间修改为2015年2月18日零时。
    \begin{enumerate}
      \item 启动计算机,以超级用户身份登录图形化用户界面。
      \item 依次选择“应用程序”$\rightarrow$“系统工具”$\rightarrow$“终端”命令,打开桌面环境下的终端工具。
      \item 输入命令\verb|date|,显示系统的当前日期和时间。
      \item 输入命令\verb|date 021800002015|,屏幕显示新修改的系统时间。在桌面环境的终端执行时显示中文提示信息。
    \end{enumerate}
  \item 切换为普通用户,查看2015年3月5日是星期几。
    \begin{enumerate}
      \item 前一操作是以超级用户身份进行的,但通常情况下只有在必须使用超级用户权限时,才以超级用户身份登录系统执行操作。为提高操作安全性,输入命令su - USER切换为普通用户USER。
      \item 输入命令\verb|cal 2015|,屏幕上显示出2015年的日历,由此可知2015年3月5日是星期四。
    \end{enumerate}
  \item 查看ls命令的-s选项的帮助信息。
    \begin{itemize}
      \item 方法一:
	\begin{enumerate}
	  \item 输入命令\verb|man ls|,屏幕显示出手册页中ls命令相关帮助信息的第一页,介绍ls命令的含义、语法结构以及-a、-A、-b和-B等选项的含义。
	  \item 使用【PgDn】键、【PgUp】键以及上、下方向键找到-s选项的说明信息。
	  \item 由此可知,ls命令的-s选项等同于\verb|--size|选项,以文件块为单位显示文件和目录的大小。
	  \item 在屏幕上的“:”后输入q,退出ls命令的手册页帮助信息。
	\end{enumerate}
      \item 方法二:
	\begin{enumerate}
	  \item 输入命令\verb|ls --help|,屏幕显示ls命令的中文帮助信息。
	  \item 拖动滚动条,找到-s选项的说明信息。
	  \item 在屏幕上的“:”后输入q,退出ls命令的手册页帮助信息。
	\end{enumerate}
    \end{itemize}
  \item 查看/etc目录下所有文件和子目录的详细信息。
    \begin{enumerate}
      \item 输入命令\verb|cd /etc|,切换到 /etc目录。
      \item 输入命令\verb|ls -al|,显示 /etc目录下所有文件和子目录的详细信息。
    \end{enumerate}
\end{enumerate}

\vspace{0.1in}
(二)字符界面下的shell命令操作
\begin{enumerate}
  \item 查看当前目录。
    \begin{enumerate}
      \item 启动计算机后,默认进入CentOS 7的图形化用户界面,此时按【Ctrl+Alt+F2】组合键切换到第二个虚拟终端。(也可使用图形化界面中的终端)
      \item 输入一个普通用户的用户名和密码,登录系统。在字符界面下输入密码时,屏幕上不会出现类似“*”的提示信息,提高了密码的安全性。
      \item 输入命令\verb|pwd|,显示当前目录。
    \end{enumerate}
  \item 用cat命令在用户主目录下创建一个名为f1的文本文件,内容为:
\begin{verbatim}
Linux is useful for us all.
You can never imagine how great it is.
\end{verbatim}
    \begin{enumerate}
      \item 输入命令\verb|cat >f1|,屏幕上输入点光标闪烁,依次输入上述内容。使用cat命令进行输入时,不能使用上、下、左、右方向键,只能用【Backspace】键来删除光标前一位置的字符。并且一旦按【Enter】键,该行输入的字符就不可修改了。
      \item 上述内容输入后,按【Enter】键,让光标处于输入内容的下一行,按【Ctrl+D】组合键结束输入。
      \item 要查看文件是否生成,输入命令\verb|ls|即可。
      \item 输入命令\verb|cat f1|,查看f1文件的内容。
    \end{enumerate}
  \item 向f1文件增加以下内容:Why not have a try?

    shell命令中可使用重定向来改变命令的执行。此处使用“\verb|>>|”符号可向文件结尾处追加内容,而如果使用“\verb|>|”符号则将覆盖已有的内容。shell命令中常用的重定向符号共3个,如下所示:
    \begin{itemize}
      \item \verb|>|:输出重定向,将前一命令执行的结果保存到某个文件。如果这个文件不存在,则创建此文件;如果这个文件已存在,则将覆盖原有内容。
      \item \verb|>>|:附加输出重定向,将前一命令执行的结果追加到某个文件。
      \item \verb|<|:将某个文件交由命令处理。
    \end{itemize}
    \begin{enumerate}
      \item 输入命令\verb|cat >>f1|,屏幕上输入点光标闪烁。
      \item 输入上述内容后,按【Enter】键,让光标处于输入内容的下一行,按【Ctrl+D】组合键结束输入。
      \item 输入命令\verb|cat f1|,查看f1文件的内容,会发现f1文件增加了一行。
    \end{enumerate}
  \item 统计f1文件的行数、单词数和字符数,并将统计结果存放到countf1文件中。
    \begin{enumerate}
      \item 输入命令\verb|wc <f1 >countf1|,屏幕上不显示任何信息。
      \item 输入命令\verb|cat countf1|,查看countf1文件的内容,其内容是f1文件的行数、单词数和字符数信息,即f1文件共有3行、19个单词和87个字符。
    \end{enumerate}
  \item 将f1和countf1文件合并为f文件。
    \begin{enumerate}
      \item 输入命令\verb|cat f1 countf1 >f|,将两个文件合并为一个文件。
      \item 输入命令\verb|cat f|,查看f文件的内容。
    \end{enumerate}
  \item 分页显示 /etc目录中所有文件和子目录的信息。

    管道符号“\verb=|=”用于连接多个命令,前一命令的输出结果是后一命令的输入内容。
    \begin{enumerate}
      \item 输入命令\verb=ls /etc | more=,屏幕显示出\verb|ls /etc|命令输出结果的第一页,屏幕的最后一行还出现“--More--”或“--更多--”字样,按【Space】键可查看下一页信息,按【Enter】键可查看下一行信息。
      \item 浏览过程中按【Q】键,可结束分页显示。
    \end{enumerate}
  \item 仅显示 /etc目录中的前5个文件和子目录。
    \begin{enumerate}
      \item 输入命令\verb=ls /etc | head -n 5=,屏幕显示出\verb|ls /etc|命令输出结果的前面5行。
    \end{enumerate}
  \item 清除屏幕内容。
    \begin{enumerate}
      \item 输入命令\verb|clear|,则屏幕内容完全被清除,命令提示符定位在屏幕左上角。
    \end{enumerate}
\end{enumerate}

\vspace{0.1in}
(三)创建链接
\begin{enumerate}
  \item 创建原文件。
    \begin{enumerate}
      \item 使用\verb|cd ~|命令定位到主目录。
      \item 使用\verb|touch original_file|命令创建一个名为original\_file的文件。
      \item 运行\verb|ls -l|命令来查看刚才创建的文件。注意该文件的链接数目。
    \end{enumerate}
  \item 创建硬链接。
    \begin{enumerate}
      \item 使用\verb|ln original_file hard_link|命令对original\_file创建一个硬链接,该链接命名为hard\_link。
      \item 再次运行\verb|ls -l|。注意文件的链接数目和上一次的修改时间。
      \item 运行\verb|ls -i|命令来显示文件的inode值。将会看到这两个文件的inode值是一样的。
    \end{enumerate}
  \item 创建软链接。
    \begin{enumerate}
      \item 使用\verb|ln -s original_file soft_link|命令为original\_file创建一个软链接,该链接命名为soft\_link。
      \item 再次使用\verb|ls -l|命令显示文件。注意文件的链接数目和修改时间。
      \item 使用\verb|ls -i|命令来查看所有文件的inode值。可以看出soft\_link的inode与original\_file和hard\_link的inode不同。
    \end{enumerate}
  \item 原文件对链接文件的影响。
    \begin{enumerate}
      \item 使用\verb|cat|命令来查看每个文件的内容,确认这些文件中没有包含任何文本或数据。
      \item 要了解原文件的改变是如何影响链接文件的,可以使用\\ \verb|echo "This text goes to the original_file" >>original_file| \\ 命令给original\_file中添加一行“This text goes to the original\_file”。
      \item 运行\verb|ls -l|命令来查看原文件和链接文件在文件大小方面的变化。虽然我们只给original\_file文件添加了数据,但hard\_link文件的大小也会发生变化,而soft\_link文件的大小没有变化。
      \item 使用\verb|cat|命令来查看文件的内容,可以发现三个文件具有完全相同的内容。
    \end{enumerate}
  \item 硬链接对原文件的影响。
    \begin{enumerate}
      \item 要了解硬链接文件的变化对原始文件的影响,可以使用\\ \verb|echo "This text goes to the hard_link file" >>hard_link| \\ 命令给hard\_link中添加一行文本“This text goes to the hrad\_link file”。
      \item 运行\verb|ls -l|并观察输出。original\_file和hard\_link的修改时间和大小都发生了改变,但是soft\_link没有变化。
      \item 现在再次使用\verb|cat|命令来显示文件的内容,注意每个文件的变化。
    \end{enumerate}
  \item 软链接对原文件的影响。
    \begin{enumerate}
      \item 如果使用\verb|echo|命令给soft\_link文件添加一行文本,将会修改原始文件,从而也更新了hard\_link文件。使用\\ \verb|echo "This text goes to the soft_link file" >>soft_link| \\ 命令给soft\_link文件中添加一行“This text goes to the soft\_link file”。
      \item 使用\verb|cat|命令来显示文件的内容。
    \end{enumerate}
\end{enumerate}

\vspace{0.1in}
(四)字符界面下使用移动存储介质
\begin{enumerate}
  \item 将光盘中的任意一个文件复制到用户主目录,最后卸载光盘。
    \begin{enumerate}
      \item 登录系统,使用命令\verb|ls /mnt/cdrom|查看光盘的挂载点是否有内容,/mnt/cdrom目录应为空。
      \item 利用mount命令,手动挂载光盘。手动挂载光盘时,不仅可以使用\verb|mount /dev/cdrom|命令,也可以使用\verb|mount /mnt/cdrom|命令。也就是说,mount命令的参数既可以是设备名,也可以是挂载点。
      \item 查看挂载点的内容,也就是查看光盘的内容。只能通过查看挂载点目录(例如,\verb|ls /mnt/cdrom|)来查看移动存储介质的内容,而\verb|ls /dev/cdrom|命令查看到的只是光盘的设备信息。
      \item 使用cp命令复制光盘中任意一个文件到用户主目录。
      \item 最后利用umount命令,卸载光盘。卸载光盘时,不仅可以使用\verb|umount /dev/cdrom|命令,也可以使用\verb|umount /mnt/cdrom|命令。
    \end{enumerate}
  \item 将U盘上的任意一个文件复制到用户主目录,并查看所有磁盘的使用情况,最后卸载U盘。
    \begin{enumerate}
      \item 登录系统,插入U盘。CentOS 7自动显示U盘的相关信息,按【Enter】键,出现命令提示符。
      \item 利用\verb|mount /mnt/usb|命令,手动挂载U盘。
      \item 使用\verb|ls /mnt/usb|命令查看U盘中的文件内容。
      \item 使用\verb|df|命令查看已挂载的文件系统,并可了解U盘的使用率。此时,不仅能查看到硬盘上分区的使用率,还能查看到U盘的挂载信息和使用率。
      \item 使用cp命令复制U盘中任意一个文件到用户主目录。
      \item 最后利用\verb|umount /mnt/usb|命令,卸载U盘。
    \end{enumerate}
\end{enumerate}

\vspace{0.1in}
(五)常见文件与文件夹操作的命令实现

回忆、总结图形界面下文件与文件夹的常见操作(如:新建、删除、重命名、移动等),在命令行界面中用命令将其实现。
