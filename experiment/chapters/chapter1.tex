\chapter{体验(虚拟机中的)Linux}

%\begin{adjustwidth}{1cm}{1cm}
%\shadowbox{
%\begin{minipage}[l]{0.8\textwidth}
{\itshape
安装Linux对于从未接触过Linux的人而言存在一定的难度。如果在已安装有其他操作系统的计算机上安装Linux,错误的设置有可能导致原有数据全部丢失,造成不可估量的损失。为熟悉Linux的安装过程,可先使用虚拟机来模拟整个安装过程。对于只想尝试Linux的用户而言,在虚拟机中安装Linux也是一个不错的选择。本实验以Ubuntu 14.10和CentOS 7为例学习Linux的安装。

虚拟机(Virtual Machine)不是一台真正的计算机,而是利用真正计算机的部分硬件资源,通过虚拟机软件模拟出一台计算机。虽然只是虚拟机,却拥有自己的CPU等外围设备。现在的虚拟机软件已经能让虚拟机的功能和真正的计算机没有什么区别。用户可以对虚拟机进行磁盘分区、格式化、安装操作系统等操作,而对本身的计算机没有任何影响。

目前,比较常用的虚拟机软件有VMware公司出品的相关产品、甲骨文公司出品的Oracle VirtualBox以及微软公司出品的Virtual PC等。本实验以VirtualBox为例说明虚拟机的使用。VirtualBox是以GNU通用公共许可证(GPL)发布的自由软件,并提供有二进制版本及开放源代码版本的代码,可从其官方网站\href{https://www.virtualbox.org/}{https://www.virtualbox.org/}下载,可运行于Windows和Linux等操作系统环境中。
}
%\end{minipage}
%}
%\end{adjustwidth}

\vspace{0.2in}
\noindent
一、实验要求
\begin{enumerate}
  \item 了解在VirtualBox/VMware中安装Linux的步骤。
  \item 启动Linux并进行初始化设置。
  \item 登录不同Linux发行版的桌面环境,了解不同发行版的图形化用户界面。
  \item 掌握注销与关机的方法。
\end{enumerate}

\vspace{0.2in}
\noindent
二、实验准备
\begin{enumerate}
  \item 一台已安装有Windows操作系统和VirtualBox/VMware软件的计算机。
  \item VirtualBox/VMware中已安装多个Linux发行版。
\end{enumerate}

\vspace{0.2in}
\noindent
三、实验内容

\vspace{0.1in}
(一)了解虚拟机和Linux发行版
\begin{enumerate}
  \item 通过检索了解虚拟机的基本概念及常用虚拟机软件。
  \item 通过检索了解常见的Linux发行版。
  \item 通过检索进入各个Linux发行版的官网,熟悉其界面。
  \item 通过检索了解Linux发行版的排名。
\end{enumerate}

为了快速、顺利完成检索,请参看以下提示:
\begin{itemize}
  \item 检索工具:引擎首选\textbf{谷歌},次选百度等;百科首选\textbf{维基百科},次选百度百科等。
  \item 检索策略:关键词首选\textbf{英文},次选中文。
  \item 发行版排名:\textbf{DistroWatch}网站。
\end{itemize}

\vspace{0.1in}
(二)Linux发行版(在虚拟机中)的安装与启动
\begin{enumerate}
  \item 在虚拟机中安装Linux(Ubuntu/CentOS等)。
\\ 检索Virtuabox/VMware的使用方法,以及在其中安装Linux的图文教程,详细阅读,理解主要的安装步骤。
  \item 启动Linux。
    \begin{enumerate}
      \item 启动Linux后,将出现用户登录列表。选择相应的用户,输入对应的密码,回车确认输入。
      \item 用户名和密码验证通过后,进入Linux桌面环境。
    \end{enumerate}
  \item 注销用户。
    \begin{enumerate}
      \item 单击右上角类似齿轮的图标,从中选择“注销…”。
      \item 在弹出的对话框中单击“注销”按钮,将退出Linux桌面环境,屏幕再次显示登录界面,等待新用户登录系统。
    \end{enumerate}
  \item 关机。
    \begin{enumerate}
      \item 单击登录界面右上角类似齿轮的图标,选择“关机…”。
      \item 在弹出的对话框中单击“关机”按钮,系统将依次停止系统的相关服务,直至完全关闭计算机。
    \end{enumerate}
\end{enumerate}

\vspace{0.1in}
(三)体验其他Linux发行版

在VirtualBox/VMware虚拟机中已经安装好了其他多个Linux发行版(Linux Mint,ElementaryOS,Fedora,openSUSE,Debian,Deepin等),依次启动这些操作系统,根据自己的喜好对各个发行版进行排名。
