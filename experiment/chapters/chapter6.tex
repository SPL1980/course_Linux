\chapter{Linux中的软件管理}

%\vspace{0.2in}
\noindent
一、实验要求
\begin{enumerate}
  \item 掌握在命令行中下载文件的方法。
  \item 掌握使用APT与Yum管理软件的方法。
  \item 掌握使用dpkg与RPM管理软件的方法。
  \item 掌握通过源代码安装软件的步骤。
  \item 熟悉其他安装软件的方法。
\end{enumerate}

\vspace{0.2in}
\noindent
二、实验准备
\begin{enumerate}
  \item 安装有Ubuntu/CentOS的计算机。
\end{enumerate}

\vspace{0.2in}
\noindent
三、实验内容

\vspace{0.1in}
(一)在命令行中下载安装包

为便于后续软件的安装,使用wget和/或curl在命令行中下载datamash、htop和Glances的安装包。通过查阅wget和curl的帮助页,掌握它们的使用方法。(提示:主要是wget的 \verb|-c| 选项和curl的 \verb|-o| 和 \verb|-O| 选项)

\vspace{0.1in}
(二)通过APT与Yum安装软件

以dos2unix(或htop,Glances等)为例,使用APT在Ubuntu中安装dos2unix,使用Yum在CentOS中安装dos2unix。
\begin{lstlisting}[language=bash]
# 需要使用sudo
apt-get install dos2unix

# 需要切换为root用户
yum install dos2unix
\end{lstlisting}

\vspace{0.1in}
(三)通过dpkg与RPM安装软件

以datamash(或Webmin,TeamViewer,RStudio等)为例,使用dpkg在Ubuntu中安装Webmin,使用RPM在CentOS中安装Webmin。
\begin{lstlisting}[language=bash]
# 需要预先使用wget或curl下载deb和/或rpm包
# 需要使用sudo
dpkg -i datamash_1.0.6-1_amd64.deb

# 需要切换为root用户
rpm -ivh datamash-1.0.6-1.el6.x86_64.rpm
\end{lstlisting}

\vspace{0.1in}
(四)通过源代码安装软件

以htop(或datamash,dos2unix,parallel,FASTX-Toolkit,msort,SAMtools,BEDTools,seqtk等)为例,使用源代码对其进行安装。
\begin{lstlisting}[language=bash]
# 需要预先使用wget或者curl下载htop的tar.gz源代码包
# htop
tar -zxvf htop-1.0.3.tar.gz
cd htop-1.0.3/
#vim INSTALL
#vim README
./configure
make
make install
\end{lstlisting}

\vspace{0.1in}
(五)通过脚本安装软件

以Glances(或cheat,Webmin等)为例,使用脚本对其进行安装。
\begin{lstlisting}[language=bash]
# 需要预先下载Glances的zip源代码包
# Glances
unzip Glances-master.zip
cd Glances-master
#vim README.rst
python setup.py install
\end{lstlisting}

\vspace{0.1in}
(六)通过其他方式安装软件

以Galaxy(或cheat,Glances等)为例,根据安装说明对其进行安装。
\begin{lstlisting}[language=bash]
# Galaxy
cd ~
git clone https://github.com/galaxyproject/galaxy
cd galaxy
git checkout -b master orgin/master
\end{lstlisting}

\vspace{0.1in}
(七)不需要安装的软件

以CPU-G(或FASTX-Toolkit,WebLogo,TeamViewer,IGV等)为例,下载后可以直接使用。
\begin{lstlisting}[language=bash]
# 需要预先下载CPU-G的tar.gz源代码包
# CPU-G
tar -zxvf cpu-g-0.9.0.tar.gz
cd cpu-g-0.9.0/
chmod 755 cpu-g
./cpu-g
\end{lstlisting}

\vspace{0.1in}
(八)软件管理

尝试使用dpkg与APT、RPM与Yum对软件(如:htop,Glances,dos2unix等)进行管理(如:查询、安装、卸载等)。

