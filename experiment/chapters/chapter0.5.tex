\chapter{远程登录Linux服务器}

%\begin{adjustwidth}{1cm}{1cm}
%\shadowbox{
%\begin{minipage}[l]{0.8\textwidth}
{\itshape
通常我们可以将服务器看做一台配置功能强大的超级电脑,它也有自己独立的操作系统,其中核心系统以Linux系统为主的服务器,我们都可以称为Linux服务器。
}
%\end{minipage}
%}
%\end{adjustwidth}

\vspace{0.2in}
\noindent
一、实验要求
\begin{enumerate}
  \item 熟悉远程登录Linux服务器的工具。
  \item 掌握远程登录Linux服务器的方法。
  \item 掌握登录和退出Linux的基本命令。
\end{enumerate}

\vspace{0.2in}
\noindent
二、实验准备
\begin{enumerate}
  \item 一台已安装有Linux操作系统的(远程)服务器。
  \item 一台安装有Windows(或/和Linux)操作系统的台式机。
\end{enumerate}

\vspace{0.2in}
\noindent
三、实验内容

\vspace{0.1in}
(一)使用Windows操作系统远程登录Linux服务器
\begin{enumerate}
  \item 检索可以使用的远程登录工具。
  \item 检索远程登录工具的使用方法并进行练习与使用。
  \item 尝试对远程登录工具进行个性化配置。
  \item 尝试与远程服务器进行文件交互。
\end{enumerate}

为了快速、顺利完成检索与学习,请参看以下提示:
\begin{itemize}
  \item 远程登录:\href{https://www.putty.org/}{PuTTY};\href{https://mobaxterm.mobatek.net/}{MobaXterm};等。
  \item 文件交互:\href{https://filezilla-project.org/}{FileZilla};等。
  \item 工具选择:首选开源工具,次选收费工具,\textbf{不要使用破解软件}。
  \item 工具下载:\textbf{一定从官网下载软件!}
  \item 检索工具:引擎首选\textbf{谷歌/必应},次选百度等;百科首选\textbf{维基百科},次选百度百科等。
  \item 检索策略:关键词首选\textbf{英文},次选中文。
\end{itemize}

\vspace{0.1in}
(二)使用Linux操作系统远程登录Linux服务器

在本地Linux操作系统中找到终端(Terminal),之后尝试登陆、退出远程Linux服务器。
\begin{enumerate}
  \item 登录命令:ssh
  \item 退出命令:logout, exit
\end{enumerate}

登录远程Linux服务器时,思考以下问题(提示:从安全的角度进行思考):
\begin{itemize}
  \item 登录前会有用户名或者密码的提示吗?
  \item 输入密码时与常规的登录网站有何不同?为什么这么“反人类”?
  \item 如果登录失败会有失败信息的提示吗?为什么这么“反人类”?
  \item 登录后看到了哪些基本信息?这些信息有什么作用?
\end{itemize}
