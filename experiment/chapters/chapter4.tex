\chapter{Linux常用命令的操作}

%\vspace{0.2in}
\noindent
一、实验要求
\begin{enumerate}
  \item 掌握联机帮助页的使用方法。
  \item 掌握命令选项的使用方法。
  \item 熟练掌握目录和文件管理的相关方法。
  \item 掌握文件权限的修改方法。
  \item 掌握文件归档和压缩的方法。
  \item 了解进行系统性能监视的基本方法。
  \item 掌握重定向、管道、通配符、历史记录等的使用方法。
  \item 掌握对文件进行排序的方法。
  \item 掌握利用shell命令管理用户和组群的方法。
  \item 理解 /etc/passwd和 /etc/group文件的含义。
  \item 了解批量新建用户账号的步骤和方法。
\end{enumerate}

\vspace{0.2in}
\noindent
二、实验准备
\begin{enumerate}
  \item 安装有CentOS 7的计算机。
\end{enumerate}

\vspace{0.2in}
\noindent
三、实验内容

\vspace{0.1in}
(一)使用联机帮助页。
\begin{enumerate}
  \item 使用联机帮助页,搜索指定的关键字,查看显示了哪些命令。如果一个关键字都想不起来,可以使用\verb|man -k shell|。
  \item 从搜索结果列表中选择一个命令,阅读它的联机帮助页。
\end{enumerate}

\vspace{0.1in}
(二)对ls命令使用选项。

尝试这个练习以了解Linux命令语法的灵活性。使用ls命令,添加-a选项以便在目录列表中包含隐藏文件;隐藏文件是那些文件名以点号开头的文件,例如.bashrc。
\begin{enumerate}
  \item 执行命令\verb|ls -l -a /etc|。
  \item 执行命令\verb|ls -la /etc|。
  \item 比较两个命令的输出。它们的结果是一样的。
\end{enumerate}

\vspace{0.1in}
(三)文件管理。
\begin{enumerate}
  \item 创建两个新目录dir1和dir2,然后将dir2目录移到dir1目录中,最后删除dir2目录。
    \begin{enumerate}
      \item 登录计算机,打开终端,当前目录为用户的主目录。
      \item 输入命令\verb|ls -l|,查看当前目录中的所有文件。
      \item 创建两个新目录,输入命令\verb|mkdir dir{1,2}|。使用mkdir命令创建多个目录时,如果目录名的开头都相同,可利用“\{ \}”符号。
      \item 再次输入命令\verb|ls -l|,确认两个目录是否成功创建。
      \item 输入命令\verb|mv dir2 dir1|,将dir2目录移动到dir1目录。
      \item 输入命令\verb|cd dir1|,切换到dir1目录,再输入ls命令,会查看到dir2目录。
      \item 输入命令\verb|rm -rf dir2|,删除dir2目录。删除目录时,当前目录不能为被删除的目录或者其子目录。
      \item 输入命令\verb|ls|,发现dir2目录确实已被删除。
      \item 输入命令\verb|cd ~|,回到用户主目录。
    \end{enumerate}
  \item 查找profile文件。
    \begin{enumerate}
      \item 由于普通用户只对部分目录具有权限,不能从所有的目录查找文件。因此,先输入命令\verb|su -|,并输入超级用户的密码,验证成功后,从普通用户切换到超级用户。
      \item 使用\verb|find / -name profile|命令进行查找,屏幕显示已找到 /etc/profile文件。
      \item 使用\verb|exit|命令,退出超级用户身份。
    \end{enumerate}
  \item 将 /etc/profile文件中所有包含HOSTNAME的行存入f4文件,并修改f4文件的权限,让所有的用户都可以读写。
    \begin{enumerate}
      \item 使用命令\verb|grep -n "HOSTNAME" /etc/profile > f4|,查找 /etc/profile文件中所有包含HOSTNAME的行,并存入f4文件。grep命令中使用“-n”选项可显示出行号。
      \item 输入命令\verb|cat f4|,查看f4文件的内容。
      \item 输入命令\verb|ls -l|,查看f4文件的详细信息。
      \item 使用\verb|chmod 666 f4|命令,修改f4文件的权限。
    \end{enumerate}
  \item 将f4文件复制到dir1目录,并在dir1目录中创建 /etc/fstab文件的符号链接文件fstab-link。
    \begin{enumerate}
      \item 输入命令\verb|cp f4 dir1|,将f4文件复制到dir1目录。
      \item 输入命令\verb|ln -s /etc/fstab fstab-link|,创建 /etc/fstab文件的符号链接文件。ln命令使用-s选项建立符号链接文件,一旦源文件被删除,符号链接文件就失效。
      \item 输入命令\verb|ls -l|,可发现淡蓝色的符号链接文件fstab-link,“\verb|->|”符号后的内容为链接文件所指向的源文件。
    \end{enumerate}
  \item 查看用户目录占用磁盘的情况。
    \begin{enumerate}
      \item 输入命令\verb|du -h|,显示当前目录和每个子目录的磁盘使用情况。
      \item 输入命令\verb|du -sh|,显示当前目录总共使用的磁盘大小。
    \end{enumerate}
\end{enumerate}

\vspace{0.1in}
(四)文件归档与压缩。
\begin{enumerate}
  \item 将 /etc/X11目录归档为X.tar文件,并将X.tar文件压缩为.gz文件。
    \begin{enumerate}
      \item 方法一。
	\begin{enumerate}
	  \item 输入命令\verb|tar -czvf X.tar.gz /etc/X11|,将 /etc/X11目录中的所有文件归档并压缩为X.tar.gz文件。
	  \item 输入命令\verb|tar -tf X.tar.gz|,可查看X.tar.gz所包含的所有文件。
	\end{enumerate}
      \item 方法二。
	\begin{enumerate}
	  \item 输入命令\verb|tar -cvf X.tar /etc/X11|,将 /etc/X11目录中的所有文件归档为X.tar文件,屏幕将显示命令的执行过程。
	  \item 输入命令\verb|ls -l *.tar|,可发现新生成一个红色的X.tar文件。
	  \item 压缩X.tar文件,输入命令\verb|gzip X.tar|。
	  \item 再次输入命令\verb|ls -l *.tar.gz|,可发现X.tar文件已被X.tar.gz文件所取代,其字节数也有所减少。
	  \item 输入命令\verb|tar -tf X.tar.gz|,查看X.tar.gz所包含的所有文件。
	  \item 为方便下一步操作,输入命令\verb=tar -tf X.tar.gz | grep applnk=,查看X.tar.gz是否打包和压缩了/etc/X11/applnk目录。
	\end{enumerate}
    \end{enumerate}
  \item 将 /etc/X11目录归档压缩为X11.tar.gz文件,但跳过 /etc/X11/applnk目录。
    \begin{enumerate}
      \item 输入命令\verb|tar --exclude /etc/X11/applnk -czvf X11.tar.gz /etc/X11|,创建新的打包压缩文件X11.tar.gz,但不包括 /etc/X11/applnk目录。打包压缩生成.tar.gz文件时,可利用“\verb|--exclude|”选项,排除不需要打包的目录或文件。
      \item 查看X11.tar.gz中是否打包和压缩了 /etc/X11/applnk目录。
    \end{enumerate}
  \item 将 /etc目录中所有2015年4月1日以后有过更新的文件,打包压缩到1504new.tar.gz文件。
    \begin{enumerate}
      \item 输入命令\verb|tar -N "2015/04/01" -czvf 1504new.tar.gz /etc|,显示大量的信息,如果在2015年4月1日以后没有更新的文件就会被跳过。打包压缩生成.tar.gz文件时,可利用“-N 时间”选项,选定指定时间以后更新的文件进行打包压缩。
    \end{enumerate}
  \item 将X.tar.gz中的 /etc/X11/Xresources文件解压缩到dir1目录。
    \begin{enumerate}
      \item 首先切换到dir1目录,也就是解压缩的目标目录。
      \item 执行\verb|tar -xzvf ~/X.tar.gz etc/X11/Xresources|命令。
      \item 查看解压缩的效果。
    \end{enumerate}
  \item 使用gzip。
    \begin{enumerate}
      \item 使用\verb|cd /tmp;touch test-file|命令在 /tmp目录中创建一个名为test-file的文件。
      \item 使用gedit或Vim在这个文件中输入一些文本——10行左右的随机语句——然后保存这个文件。
      \item 使用\verb|ls -l|命令显示文件的大小。
      \item 为了证明可以查看这个文件,因为它是一个普通的文本文件,使用cat命令。
      \item 使用gzip压缩这个刚才创建的文件。
      \item 现在在 /tmp目录中产生了一个名为test-file.gz的文件。
      \item 使用\verb|ls -l|命令查看压缩文件的大小。这个文件应该比未压缩的版本占用较少的空间,因为gzip压缩了它的大小。
      \item 使用gzip或gunzip解压文件:\verb|gzip -d test-file.gz|或\verb|gunzip -d test-file.gz|。
      \item 现在/tmp目录中有一个名为test-file的文件(test-file.gz文件已不存在)。使用cat命令查看该文件的内容。
      \item 看到的输出应该和压缩该文件前使用cat命令看到的一样。再次使用\verb|ls -l|命令,将看到这个文件和原来的文件大小相同。
    \end{enumerate}
\end{enumerate}

\vspace{0.1in}
(五)利用shell监视系统性能。
\begin{enumerate}
  \item 输入命令top,屏幕动态显示CPU使用率、内存使用率和进程状态等相关信息,且默认以CPU使用率进行排列。
  \item 按【M】键,所有进程按照内存使用率排列。
  \item 按【T】键,所有进程按照执行时间排列。
  \item 按【P】键,恢复按照CPU使用率排列所有进程。
  \item 按【Ctrl+C】组合键结束top命令。
\end{enumerate}

\vspace{0.1in}
(六)通配符的使用。

shell命令的通配符包括\verb|*|、\verb|?|、\verb|[]|、\verb|-|和\verb|!|,灵活使用通配符可同时引用多个文件,方便操作。
\begin{itemize}
  \item \verb|*|:匹配任意长度的任何字符。
  \item \verb|?|:匹配一个字符。
  \item \verb|[]|:表示范围。
  \item \verb|-|:通常与\verb|[]|配合使用,起始字符-终止字符构成范围。
  \item \verb|!|:表示不在范围,通常也与\verb|[]|配合使用。
\end{itemize}
\begin{enumerate}
  \item 显示 /bin目录中所有以c为首字母的文件和目录。
    \begin{enumerate}
      \item 输入命令\verb|ls /bin/c*|,屏幕将显示 /bin目录中以c开头的所有文件和目录。
    \end{enumerate}
  \item 显示 /bin目录中所有以c为首字母、文件名只有3个字符的文件和目录。

    shell可以记录一定数量的已执行过的命令,当用户需要再次执行时,不用再次输入,可以直接调用。使用上、下方向键,【PaUp】或【PgDn】键,在shell命令提示符后将出现已执行过的命令。直接按【Enter】键就可以再次执行这一命令,也可以对出现的命令行进行编辑,修改为用户所需要的命令后再执行。
    \begin{enumerate}
      \item 按向上方向键,shell命令提示符后出现上一步操作时输入的命令\verb|ls /bin/c*|。
      \item 将其修改为\verb|ls /bin/c??|并执行,屏幕显示 /bin目录中以c为首字母、文件名只有3个字符的文件和目录。
    \end{enumerate}
  \item 显示 /bin目录中所有的首字母为c或s或h的文件和目录。
    \begin{enumerate}
      \item 输入命令\verb|ls /bin/[csh]*|,屏幕显示 /bin目录中首字母为c或s或h的文件和目录。\verb|[csh]*|并非表示所有以csh开头的文件,而是表示以c或s或h为首字母的文件。为避免误解,也可以使用\verb|[c,s,h]*|,达到相同的效果。
    \end{enumerate}
  \item 显示 /bin目录中所有首字母是v、w、x、y、z的文件和目录。
    \begin{enumerate}
      \item 输入命令\verb|ls /bin/[!a-u]*|,屏幕显示 /bin目录中首字母是v~z的文件和目录。
    \end{enumerate}
  \item 重复上一步操作。
    
    用户不仅可利用上、下方向键来显示执行过的命令,还可以使用history命令查看或调用执行过的命令。history命令可以查看到已执行命令在历史记录列表中的序号,使用“!序号”命令即可进行调用,而“\verb|!!|”命令则执行最后执行过的那个命令。
    \begin{enumerate}
      \item 输入命令\verb|!!|,自动执行上一步操作中使用过的\verb|ls /bin/[!a-u]*|命令。
    \end{enumerate}
  \item 查看刚执行过的5个命令。
    \begin{enumerate}
      \item 输入命令\verb|history 5|,显示最近执行过的5个命令。
    \end{enumerate}
\end{enumerate}

\vspace{0.1in}
(七)对文件进行排序。
\begin{enumerate}
  \item 用下面的文本创建文件 /tmp/outoforder:
\begin{verbatim}
Zebra
Quebec
hosts
Alpha
Romeo
juliet
unix
XRay
Xray
Sierra
Charlie
horse
horse
horse
Bravo
1
11
2
23
\end{verbatim}
  \item 以字典的顺序排列文件:\verb|sort -d /tmp/outoforder|。注意,以大写字母开头的字符串位于所有小写单词的前面。
  \item 单词horse在文件中出现了3次。要去冗余,即删除排序中额外的实例,可以使用如下命令:\verb|sort -du /tmp/outoforder|。
\end{enumerate}

\vspace{0.1in}
(八)利用shell命令管理用户与组。
\begin{enumerate}
  \item 新建一名为duser的用户,其密码是tdd63u2,主要组群为myusers。
    \begin{enumerate}
      \item 按【Ctrl+Alt+F3】组合键,切换到第三个虚拟终端,以超级用户身份登录。
      \item 输入命令\verb|useradd -g myusers duser|,建立新用户duser,其主要组群是myusers。
      \item 为新用户设置密码,输入命令\verb|passwd duser|,根据屏幕提示输入两次密码,最后屏幕提示密码成功设置信息。设置用户密码时,输入的密码在屏幕上并不显示出来,而输入两次的目的在于确保密码没有输错。
      \item 输入命令\verb|cat /etc/passwd|,查看 /etc/passwd文件的内容,发现文件的末尾增加了duser用户的信息。
      \item 输入命令\verb|cat /etc/group|,查看 /etc/group文件的内容,发现文件内容并未增加。
      \item 按【Ctrl+Alt+F4】组合键,切换到第四个虚拟终端,输入duser用户名和密码可登录系统。
      \item 输入命令\verb|exit|,duser用户退出登录。
    \end{enumerate}
  \item 将duser用户设置为不需要密码就能登录。
    \begin{enumerate}
      \item 按【Ctrl+Alt+F3】组合键,切换到正被超级用户使用的第三个虚拟终端。
      \item 输入命令\verb|passwd -d duser|。
      \item 按【Ctrl+Alt+F4】组合键,再次切换到第四个虚拟终端,在“Login:”后输入用户名duser,按【Enter】键直接出现shell命令提示符,说明duser用户不需要密码即可登录。
    \end{enumerate}
  \item 查看duser用户的相关信息。
    \begin{enumerate}
      \item 在第三个虚拟终端输入命令\verb|id duser|,显示duser用户的用户ID(UID)、主要组群的名称和ID(GID)。
    \end{enumerate}
  \item 从普通用户duser切换为超级用户。
    \begin{enumerate}
      \item 第四个虚拟终端当前的shell命令提示符为“\verb|$|”,表明当前用户是普通用户。
      \item 输入命令\verb|ls /root|,屏幕上未列出 /root目录中的文件和子目录,而是出现提示信息,提示当前用户没有查看 /root目录的权限。
      \item 输入命令\verb|su -| 或者是\verb|su - root|,屏幕提示输入密码,此时输入超级用户的密码,验证成功后shell提示符从“\verb|$|”变为“\verb|#|”,说明已从普通用户转换为超级用户。
      \item 再次输入命令\verb|ls /root|,可查看 /root目录中的文件和子目录。
      \item 输入命令\verb|exit|,回到普通用户的工作状态。
      \item 输入命令\verb|exit|,duser用户退出登录。
    \end{enumerate}
  \item 一次性删除duser用户及其工作目录。
    \begin{enumerate}
      \item 按【Ctrl+Alt+F3】组合键,切换到正被超级用户使用的第三个虚拟终端。
      \item 输入命令\verb|userdel -r duser|,删除duser用户。处于登录状态的用户不能删除。如果在新建这个用户时还创建了私人组,而该私人组当前又没有其他用户,那么在删除用户的同时也将一并删除这一私人组。
      \item 输入命令\verb|cat /etc/passwd|,查看 /etc/passwd文件的内容,发现duser用户的相关信息已消失。
      \item 输入命令\verb|ls /home|,发现duser用户的主目录 /home/duser已不复存在。
    \end{enumerate}
  \item 新建组mygroup。
    \begin{enumerate}
      \item 在超级用户的shell提示符后输入命令\verb|groupadd mygroup|,建立mygroup组。
      \item 输入命令\verb|cat /etc/group|,发现group文件的末尾出现mygroup组的信息。
      \item 输入命令\verb|cat /etc/gshadow|,发现gshadow文件的末尾也出现mygroup组的信息。
    \end{enumerate}
  \item 将mygroup组改名为newgroup。
    \begin{enumerate}
      \item 在超级用户的shell提示符后输入命令\verb|groupmod -n newgroup mygroup|,其中-n选项表示更改组名称。
      \item 输入命令\verb|cat /etc/group|,查看组信息,发现原来mygroup所在行的第一项变为newgroup。
    \end{enumerate}
  \item 删除newgroup组。
    \begin{enumerate}
      \item 超级用户输入命令\verb|groupdel newgroup|,删除newgroup组。
    \end{enumerate}
\end{enumerate}

\vspace{0.1in}
(九)批量新建多个用户账号。
\begin{enumerate}
  \item 为全班30位同学创建用户账号,用户名为“s”+学号的组合,其中班级名册中第一位同学的学号为140101。所有同学都属于class1401组。所有同学的初始密码都为123456。
    \begin{enumerate}
      \item 以超级用户身份登录系统,输入命令\verb|groupadd -g 600 class1401|(假设值为600的GID未被使用),新建全班同学的组class1401。
      \item 输入命令vim students,新建用户信息文件。
      \item 按【i】键,切换为Vim的文本编辑模式,输入第一行信息“s140101:x:601:600::/home/s140101:/bin/bash”,完成后按【Esc】键,切换到命令模式。
      \item 按【:】键,并输入\verb|1,1co1|。此时将本身的一行复制为两行,继续输入\verb|1,2co2|从两行复制为四行,依此类推,直到出现30行信息。
      \item 按【i】键,切换为文本编辑模式,修改每位同学用户信息不同的部分,编辑完成的文件如下所示。最后保存并退出Vim。
\begin{verbatim}
s140101:x:601:600::/home/s140101:/bin/bash
s140102:x:602:600::/home/s140102:/bin/bash
s140103:x:603:600::/home/s140103:/bin/bash
...
\end{verbatim}
      \item 输入命令vim stu-passwd,新建用户密码文件。
      \item 按【i】键,切换为Vim的文本编辑模式,输入第一行信息“s140101:123456”,即所有同学的初始密码为123456,然后进行复制,直至复制出30行信息,然后依次修改用户名。完成的文件如下所示,保存并退出Vim。
\begin{verbatim}
s140101:123456
s140102:123456
s140103:123456
...
\end{verbatim}
      \item 输入命令\verb|newusers < students|,批量新建用户账号。
      \item 输入命令\verb|pwunconv|,暂时取消shadow加密。
      \item 输入命令\verb|chpasswd < stu-passwd|,批量新建用户的密码。
      \item 输入命令\verb|pwconv|,进行shadow加密,完成批量创建用户账号工作。
      \item 输入命令\verb|cat /etc/passwd|,查看/etc/passwd文件将发现所有的用户账号均已建立。
      \item 可尝试以新建的用户名登录,并应该及时修改用户的密码。
      \item 使用此方法批量建立的用户登录系统时,其命令提示符不是默认的格式:\verb|[用户名@localhost ~]$|,而是\verb|-bash-4.1$|。如果希望使用标准的命令提示符,可复制用户主目录中的配置文件.bash\_profile和.bahsrc到那些批量创建的用户的主目录中,再次登录,恢复标准的shell命令提示符。
    \end{enumerate}
\end{enumerate}

