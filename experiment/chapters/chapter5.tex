\chapter{Linux高级命令的操作}

%\vspace{0.2in}
\noindent
一、实验要求
\begin{enumerate}
  \item 掌握find的使用方法。
  \item 掌握grep的使用方法。
  \item 掌握sed的使用方法。
  \item 掌握AWK的使用方法。
  \item 了解Linux命令在生物信息学文本处理中的应用。
\end{enumerate}

\vspace{0.2in}
\noindent
二、实验准备
\begin{enumerate}
  \item 安装有Ubuntu/CentOS的计算机。
\end{enumerate}

\vspace{0.2in}
\noindent
三、实验内容

\vspace{0.1in}
(一)使用find
\begin{enumerate}
  \item 使用以下命令,创建测试find命令的文件。
\begin{lstlisting}[language=bash]
# 切换至主目录
cd ~

# 创建测试find命令的目录
mkdir find_test

# 切换至该目录
cd find_test

# 创建空文件
touch MybashProgram.sh
touch mycprogram.c
touch MyCProgram.c
touch Program.c

# 创建文件夹和文件 
mkdir backup
cd backup
touch MybashProgram.sh
touch mycprogram.c
touch MyCProgram.c
touch Program.c
 
# 切换回目录
cd ..

# 检查文件夹和文件的创建结果
ls -R
\end{lstlisting}
  \item 用文件名查找文件。
    \begin{enumerate}
      \item 用MyCProgram.c作为查找名在当前目录及其子目录中查找文件:\\ \verb|find -name "MyCProgram.c"|。
      \item 用MyCProgram.c作为查找名在当前目录及其子目录中查找文件,忽略大小写:\\ \verb|find -iname "MyCProgram.c"|。
    \end{enumerate}
  \item 在find命令查找到的文件上执行命令。
    \begin{enumerate}
      \item 计算所有不区分大小写的文件名为“MyCProgram.c”的文件的MD5校验和:\\ \verb|find -iname "MyCProgram.c" -exec md5sum {} \;|。\verb|{}|将会被当前文件名取代。
    \end{enumerate}
  \item 相反匹配。
    \begin{enumerate}
      \item 显示所有名字不是MyCProgram.c的文件或者目录:\\ \verb|find -maxdepth 1 -not -iname "MyCProgram.c"|。由于maxdepth是1,所以只会显示当前目录下的文件和目录。
    \end{enumerate}
  \item 限定搜索指定目录的深度。
    \begin{enumerate}
      \item 在root目录及其子目录下查找passwd文件:\verb|find / -name passwd|。
      \item 在root目录及其1层深的子目录中查找passwd:\verb|find / -maxdepth 2 -name passwd|。
      \item 在root目录下及其最大两层深度的子目录中查找passwd文件:\\ \verb|find / -maxdepth 3 -name passwd|。
      \item 在第二层子目录和第四层子目录之间查找passwd文件:\\ \verb|find / -mindepth 3 -maxdepth 5 -name passwd|。
    \end{enumerate}
  \item 使用inode编号查找文件。
    \begin{enumerate}
      \item 创建两个名字相似的文件,例如一个有空格结尾,一个没有:\\ \verb|touch "test-file-name"|,\verb|touch "test-file-name "|。
      \item \verb|ls -l test*|。从ls的输出不能区分哪个文件是以空格结尾的。使用选项 \verb|-i|,可以看到文件的inode编号,借此可以区分这两个文件:\verb|ls -il test*|。
      \item 用inode编号重命名文件名中有特殊符号的文件:\\ \verb|find -inum INODE -exec mv {} new-test-file-name \;|,\\ 或者删除它:\verb|find -inum INODE -exec rm {} \;|。
    \end{enumerate}
  \item 根据文件权限查找文件。
    \begin{enumerate}
      \item 找到当前目录下对同组用户具有读权限的文件,忽略该文件的其他权限:\\ \verb|find . -perm -g=r -type f -exec ls -l {} \;|。
      \item 找到对组用户具有只读权限的文件:\\ \verb|find . -perm g=r -type f -exec ls -l {} \;|。(仔细比较该命令和上一个命令的区别)
      \item 找到对组用户具有只读权限的文件(使用八进制权限形式):\\ \verb|find . -perm 040 -type f -exec ls -l {} \;|。
    \end{enumerate}
  \item 查找空文件。
    \begin{enumerate}
      \item 找到home目录及子目录下所有的空文件(0字节文件):\verb|find ~ -empty|。
      \item 只列出home目录里的空文件:\verb|find ~ -maxdepth 1 -empty|。
      \item 只列出当前目录下的非隐藏空文件:\\ \verb|find . -maxdepth 1 -empty -not -name ".*"|。
    \end{enumerate}
  \item 查找最大最小的文件。
    \begin{enumerate}
      \item 列出当前目录及子目录下5个最大的文件:\\ \verb=find . -type f -exec ls -s {} \; | sort -n -r | head -5=。
      \item 查找5个最小的文件:\\ \verb=find . -type f -exec ls -s {} \; | sort -n  | head -5=。
      \item 列出最小的文件,而不是0字节文件:\\ \verb=find . -not -empty -type f -exec ls -s {} \; | sort -n  | head -5=。
    \end{enumerate}
  \item 查找指定文件类型的文件。
    \begin{enumerate}
      \item 查找socket文件:\verb|find . -type s|。
      \item 查找所有的目录:\verb|find . -type d|。
      \item 查找所有的一般文件:\verb|find . -type f|。
      \item 查找所有的隐藏文件:\verb|find . -type f -name ".*"|。
      \item 查找所有的隐藏目录:\verb|find . -type d -name ".*"|。
    \end{enumerate}
  \item 通过文件大小查找文件。\verb|+| 指比给定尺寸大,\verb|-| 指比给定尺寸小,没有符号代表和给定尺寸完全一样大。 
    \begin{enumerate}
      \item 查找比指定文件大的文件:\verb|find ~ -size +100M|。
      \item 查找比指定文件小的文件:\verb|find ~ -size -100M|。
      \item 查找符合给定大小的文件:\verb|find ~ -size 100M|。
    \end{enumerate}
  \item 删除大型打包文件\textcolor{red}{【谨慎操作,请勿尝试】}。
    \begin{enumerate}
      \item 删除大于100M的*.zip文件:\\ \verb|find / -type f -name *.zip -size +100M -exec rm -i {} \;|。
      \item 删除所有大于100M的*.tar文件:\\ \verb|find / -type f -name *.tar -size +100M -exec rm -i {} \;|。
    \end{enumerate}
  \item 基于访问/修改/更改时间查找文件。
    \begin{enumerate}
      \item 找到当前目录及其子目录下,最近一次修改时间在1个小时(60分钟)之内的文件或目录:\verb|find . -mmin -60|。
      \item 找到24小时(1天)内被修改过的文件(文件系统根目录/下):\verb|find / -mtime -1|。
      \item 找到当前目录及其子目录下,最近一次访问时间在1个小时(60分钟)之内的文件或目录:\verb|find . -amin -60|。
      \item 找到24小时(1天)内被访问过的文件(文件系统根目录/下):\verb|find / -atime -1|。
      \item 在当前目录及其子目录下,查找1个小时(60分钟)内文件状态发生改变的文件:\\ \verb|find . -cmin -60|。
      \item 在根目录/及其子目录下,查找1天(24小时)内文件状态发生改变的文件:\\ \verb|find / -ctime -1|。
      \item 显示30分钟内被修改过的文件,但文件夹不显示:\\ \verb|find /etc/sysconfig -mmin -30 -type f|。
      \item 显示当前目录及其子目录下,15分钟内文件内容被修改过的文件,并且只列出非隐藏文件:\\ \verb|find . -mmin -15 \( ! -regex ".*/\..*" \)|。
    \end{enumerate}
  \item 基于文件比较的查找。
    \begin{enumerate}
      \item 显示所有在ordinary\_file之后创建修改的文件:\verb|find -newer ordinary_file|。
      \item 显示在 /etc/passwd修改之后被修改过的文件:\verb|find -newer /etc/passwd|。
      \item 显示所有在 /etc/hosts文件被修改之后被访问过的文件:\verb|find -anewer /etc/hosts|。
      \item 显示在修改文件 /etc/fstab之后所有文件状态发生改变的文件:\verb|find -cnewer /etc/fstab|。
    \end{enumerate}
  \item 在查找到的文件列表结果上直接执行命令。
    \begin{enumerate}
      \item 在find命令输出上使用 ls -l,列举出1小时内被编辑过的文件的详细信息:\\ \verb|find -mmin -60 -exec ls -l {} \;|。
      \item 在同一个命令中使用多个{}:\verb|find -name "*.txt" -exec cp {} {}.bak \;|。
      \item 将所有mp3文件的文件名中的空格替换成下划线:\\ \verb|find . -type f -iname “*.mp3″ -exec rename “s/ /_/g” {} \;|。
    \end{enumerate}
  \item 将错误重定向到 /dev/null。
    \begin{enumerate}
      \item 将错误消息重定向到 /dev/null中去:\verb|find / -name "*.conf" 2>>/dev/null|。
    \end{enumerate}
\end{enumerate}

\vspace{0.1in}
(二)使用grep
\begin{enumerate}
  \item 搜索和寻找文件:\verb=sudo dpkg -l | grep -i python=。
  \item 搜索和过滤文件,除掉所有的注释行:\verb|grep -v "#" /etc/apache2/sites-available/default-ssl|。
  \item 找出艺术家JayZ的所有mp3格式的音乐文件,里面也不要有任何混合音轨:\\ \verb=find . -name ".mp3" | grep -i JayZ | grep -vi "remix"=。
  \item 显示搜索字符串后面、前面或前后的几行:\\ \verb=ifconfig | grep -A 4 UP=,\verb=ifconfig | grep -B 2 UP|=,\verb=ifconfig | grep -C 2 lo=。
  \item 计算匹配项的数目:\verb=ifconfig | grep -c inet6=。
  \item 按给定字符串搜索文件中匹配的行号:\verb|grep -n "main" setup.py|。
  \item 递归搜索:\verb|grep -r "function" *|。
  \item 进行精确匹配搜索(包含要搜索的单词,而不是通配):\verb=ifconfig | grep -w RUNNING=。
  \item 在Gzip压缩文件中搜索:\verb|zgrep -i error /var/log/syslog.2.gz|。
  \item 在一个文件或文件列表中搜索固定样式的字符串(读取保存在文件中的匹配模式),功能与grep -F相同:\verb|fgrep -f file_full_of_patterns.txt file_to_search.txt|。
\end{enumerate}

\vspace{0.1in}
(三)使用sed进行操作
\begin{enumerate}
  \item 使用vim创建两个文件,每个文件含有一组名字:
\begin{verbatim}
#names1.txt
Paul
Craig
Debra
Joe
Jeremy

#names2.txt
Paul
Katie
Mike
Tom
Pat
\end{verbatim}
  \item 在命令行中输入并运行下面的命令:\\ \verb|sed -e s/Paul/Pablo/g names1.txt names2.txt > names3.txt|。
  \item 使用 \verb|cat names3.txt| 命令显示第三个文件的输出,以观察得到的名字列表。
\end{enumerate}

\vspace{0.1in}
(四)使用带有多个命令的sed
\begin{enumerate}
  \item 定位先前示例中创建的两个含有一组名字的文本文件。
  \item 利用vim命令创建一个名为edits.sedscr的新文件,并为sed列出一组编辑指令:
\begin{verbatim}
s/Pat/Patricia/
s/Tom/Thomas/
s/Joe/Joseph/
1d
\end{verbatim}
  \item 调用sed:\verb|sed -f edits.sedscr names1.txt names2.txt > names3.txt|。查看names3.txt文件中的内容。
\end{enumerate}

\vspace{0.1in}
(五)使用AWK
\begin{enumerate}
  \item 在命令行中输入如下的awk命令:\verb|awk '{print $0}' /etc/passwd|。
\end{enumerate}

\vspace{0.1in}
(六)使用AWK文件
\begin{enumerate}
  \item 使用vim命令输入如下内容,并将文件保存为print.awk:
\begin{verbatim}
BEGIN {
  FS=":"
}
{ printf "username: " $1 "\t\t\t user id: " $3 "\n"}
\end{verbatim}
  \item 以如下方式执行awk命令:\verb|awk -f print.awk /etc/passwd|。
\end{enumerate}

\vspace{0.1in}
(七)Linux命令在生物信息学文本处理中的应用
\begin{enumerate}
  \item Useful bash one-liners useful for bioinformatics:\\ \href{https://github.com/stephenturner/oneliners}{https://github.com/stephenturner/oneliners}。
  \item Unix commands applied to bioinformatics:\\ \href{http://rous.mit.edu/index.php/Unix\_commands\_applied\_to\_bioinformatics}{http://rous.mit.edu/index.php/Unix\_commands\_applied\_to\_bioinformatics}。
  \item Scott's list of linux one-liners:\\ \href{https://wikis.utexas.edu/display/bioiteam/Scott\%27s+list+of+linux+one-liners}{https://wikis.utexas.edu/display/bioiteam/Scott\%27s+list+of+linux+one-liners}。
\end{enumerate}

\vspace{0.1in}
(八)【课外扩展】使用datamash
\begin{enumerate}
  \item datamash与AWK、Perl和R的比较(\href{http://www.gnu.org/software/datamash/alternatives/}{http://www.gnu.org/software/datamash/alternatives/})。
    \begin{enumerate}
      \item 计算总和、最小值、最大值、平均值:
\begin{lstlisting}[language=bash]
# sum
seq 10 | datamash sum 1
seq 10 | awk '{sum+=$1} END {print sum}'

# minimum value
seq -5 1 7 | datamash min 1
seq -5 1 7 | awk 'NR==1 {min=$1} NR>1 && $1<min { min=$1  } END {print min}'

# maximum value
seq -5 -1 | datamash max 1
seq -5 -1 | awk 'NR==1 {max=$1} NR>1 && $1>max { max=$1  } END {print max}'

# mean
seq 10 | datamash mean 1
seq 10 | awk '{sum+=$1} END {print sum/NR}'
\end{lstlisting}
      \item 处理分组数据:
\begin{lstlisting}[language=bash]
# data
DATA=$(printf "%s\t%d\n" a 1 b 2 a 3 b 4 a 3 a 6)

# First value of each group
echo "$DATA" | datamash -s -g 1 first 2
echo "$DATA" | awk '!($1 in a){a[$1]=$2} END {for(i in a) { print i, a[i] }}'

# Last value of each group:
echo "$DATA" | datamash -s -g 1 last 2
echo "$DATA" | awk '{a[$1]=$2} END {for(i in a) { print i, a[i]  }}'

# Number of values in each group:
echo "$DATA" | datamash -s -g 1 count 2
echo "$DATA" | awk '{a[$1]++} END {for(i in a) { print i, a[i]  }}'

# Collapse all values in each group:
echo "$DATA" | datamash -s -g 1 collapse 2
echo "$DATA" | perl -lane '{push @{$a{$F[0]}},$F[1]} END{print join("\n",map{"$_ ".join(",",@{$a{$_}})} sort keys %a);}'

# Collapse unique values in each group:
echo "$DATA" | datamash -s -g 1 unique 2
echo "$DATA" | perl -lane '{$a{$F[0]}{$F[1]}=1} END{print join("\n",map{"$_ ".join(",",sort keys %{$a{$_}})} sort keys %a);}'

# Print a random value from each group:
echo "$DATA" | datamash -s -g 1 rand 2
echo "$DATA" | perl -lane '{ push @{$a{$F[0]}},$F[1]  } END{ print join("\n",map{"$_ ".$a{$_}->[rand(@{$a{$_}})] } sort keys %a ) ; }'
\end{lstlisting}
      \item 统计操作:
\begin{lstlisting}[language=bash]
# A simple summary of the data, without grouping:
echo "$DATA" | datamash min 2 q1 2 median 2 mean 2 q3 2 max 2
echo "$DATA" | Rscript -e 'summary(read.table("stdin"))

# A simple summary of the data, with grouping:
echo "$DATA" | datamash -s --header-out -g 1 min 2 q1 2 median 2 mean 2 q3 2 max 2 | expand -t 18
echo "$DATA" | Rscript -e 'a=read.table("stdin")' -e 'aggregate(a$V2,by=list(a$V1),summary)'

# Calculating mean and standard-deviation for each group:
echo "$DATA" | datamash -s -g 1 mean 2 sstdev 2
echo "$DATA" | Rscript -e 'a=read.table("stdin")' -e 'f=function(x){c(mean(x),sd(x))}' -e 'aggregate(a$V2,by=list(a$V1),f)'
\end{lstlisting}
      \item 反转、转置:
\begin{lstlisting}[language=bash]
# reverse fields:
echo "$DATA" | datamash reverse
echo "$DATA" | perl -lane 'print join(" ", reverse @F)'

# transpose a file (swap rows and columns):
echo "$DATA" | datamash transpose
echo "$DATA" | Rscript -e 'write.table(t(read.table("stdin")),quote=F,col.names=F,row.names=F)'
\end{lstlisting}
    \end{enumerate}
  \item datamash在生物信息学文本处理中的应用(\href{http://www.gnu.org/software/datamash/examples/\#example\_genes}{http://www.gnu.org/software/datamash/examples/\#example\_genes})。
    \begin{enumerate}
      \item 使用的数据:\href{http://git.savannah.gnu.org/cgit/datamash.git/plain/examples/genes.txt}{http://git.savannah.gnu.org/cgit/datamash.git/plain/examples/genes.txt}。
      \item 数据中16列的含义:\circled{1}bin,\circled{2}name (isoform/transcript identifier),\circled{3}chromosome,\circled{4}strand,\circled{5}txStart (transcription start site),\circled{6}txEnd (transcription end site),\circled{7}cdsStart (coding start site),\circled{8}cdsEnd (coding end site),\circled{9}exonCount n(umber of exons),\circled{10}exonStarts,\circled{11}exonEnds,\circled{12}score,\circled{13}GeneName (gene identifier),\circled{14}cdsStartStat,\circled{15}cdsEndStat,\circled{16}exonFrames。
      \item 使用datamash处理genes.txt文件:
\begin{lstlisting}[language=bash]
# Find Number of isoforms per gene
datamash -s -g 13 count 2 < genes.txt
# print all the isoforms for each gene
datamash -s -g 13 count 2 collapse 2 < genes.txt

# Find genes with more than 5 isoforms
cat genes.txt | datamash -s -g 13 count 2 collapse 2 | awk '$2>5'

# Which genes are transcribes from both strands? 
cat genes.txt | datamash -s -g 13 countunique 4 | awk '$2>1'

# Which genes are transcribes from multiple chromosomes? 
cat genes.txt | datamash -s -g 13 countunique 3 unique 3 | awk '$2>1'

# Examine Exon-count variability
cat genes.txt | datamash -s -g 13 count 9 min 9 max 9 mean 9 pstdev 9 | awk '$2>1'

# How many transcripts are in each chromosome?
datamash -s -g 3 count 2 < genes.txt

# How many transcripts are in each chromsomse AND strand?
datamash -s -g 3,4 count 2 < genes.txt
\end{lstlisting}
    \end{enumerate}
\end{enumerate}

