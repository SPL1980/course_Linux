\chapter{Vim编辑器的基本操作}

%\vspace{0.2in}
\noindent
一、实验要求
\begin{enumerate}
  \item 熟悉并掌握Vim三种工作模式之间的转换方法。
  \item 掌握新建和保存文件的操作方法。
  \item 掌握插入和删除文本的操作方法。
  \item 掌握查找和替换字符串的操作方法。
  \item 掌握Vim的常用操作,如移动定位、复制粘贴、修改删除和重做撤销等。
\end{enumerate}

\vspace{0.2in}
\noindent
二、实验准备
\begin{enumerate}
  \item 安装有Vim编辑器的计算机。
\end{enumerate}

\vspace{0.2in}
\noindent
三、实验内容

\vspace{0.1in}
(一)Vim命令概览
    \begin{enumerate}
      \item 使用 \verb|vim /tmp/beginning_unix_testfile| 在 /tmp目录下创建一个新文件,命名为beginning\_unix\_testfile。
      \item 目前处在命令模式下,通过输入【i】切换到插入模式。
      \item 输入下面的内容,在每行的末尾都必须按【Enter】键:
\begin{verbatim}
The quick brown fox jumps over the lazy dog.
Sentence two.
Sentence three.
Sentence four.
Vi will never become vii.
Sentence six.
\end{verbatim}
      \item 按下【Esc】键返回到命令模式。
      \item 输入 \verb|1G| 使游标移动到第一行,然后输入 \verb|4l|,从而使游标放在quick中的q上。
      \item 输入 \verb|cw|,该命令将删除单词quick并切换到插入模式。
      \item 输入单词slow,然后按下【Esc】键。文件现在的内容是:
\begin{verbatim}
The slow brown fox jumps over the lazy dog.
Sentence two.
Sentence three.
Sentence four.
Vi will never become vii.
Sentence six.
\end{verbatim}
      \item 输入 \verb|2j| 使游标向下移动两行。游标将位于第3行上的Sentence中的最后一个e上。
      \item 输入 \verb|r|,然后输入 \verb|E|。该句将变成:
\begin{verbatim}
SentencE three.
\end{verbatim}
      \item 输入 \verb|k| 使游标上移一行。输入2yy复制两行。
      \item 输入 \verb|4j| 使游标移动到最后一行,然后输入 \verb|p| 粘贴缓冲区内的文本。得到的结果是:
\begin{verbatim}
The slow brown fox jumps over the lazy dog.
Sentence two.
SentencE three.
Sentence four.
Vi will never become vii.
Sentence six.
Sentence two.
SentencE three.
\end{verbatim}
      \item 按下【Esc】键,然后输入 \verb|:q!| 退出该文件(!没有保存输入!)。
    \end{enumerate}

\vspace{0.1in}
(二)新建文本文件
\begin{enumerate}
  \item 利用Vim新建文件f3,内容为:
\begin{verbatim}
How to Read Faster

When I was a schoolboy I must have read every comic book ever published.
But as I got older, my eyeballs must have slowed down or something I
mean, comic books started to pile up faster than I could read them!

It wasn't until much later, when I was studying at college, I realized
that it wasn't my eyeballs that had gone wrong. They're still moving as
well as ever. The problem is that there's too much to read these days,
and too little time to read every WORD of it.
\end{verbatim}
    \begin{enumerate}
      \item 启动计算机,以普通用户身份登录字符界面。
      \item 在Shell命令提示符后输入命令 \verb|vim|,启动Vim文本编辑器。此时,Vim默认进入命令模式。
      \item 按【i】键,从命令模式转换为文本编辑模式,此时屏幕的最底边出现“--INSERT--”字样。
      \item 输入上述文本内容。输入过程中如果出错,可使用【Backspace】键或【Delete】键删除错误的字符。
      \item 输入完成后,按【Esc】键返回命令模式。
      \item 按【:】键进入末行模式,输入 \verb|w f2|,将正在编辑的内容保存为f2文件。
      \item 屏幕底部显示“``f2'' [New] 3L, 495C written”字样,表示此文件有3行、495个字符。\textbf{知识点解析:}Vim中行的概念与平时所说的行有所区别,在输入文字的过程中由于字符串长度超过屏幕宽度而发生的自动换行,Vim并不认为是新的一行,只有在Vim中按一次【Enter】键、另起一行的才算作新的一行。
      \item 按【:】键后输入 \verb|q|,退出Vim。
    \end{enumerate}
\end{enumerate}

\vspace{0.1in}
(三)编辑文件
\begin{enumerate}
  \item 打开f2文件并显示行号。
    \begin{enumerate}
      \item 输入命令 \verb|vim f2|,启动Vim文本编辑器并打开f2文件。
      \item 按【:】键切换到末行模式,输入 \verb|set nu|,每一行前出现行号。
      \item Vim自动返回到命令模式,连续两次按下【Z】键(注意大写),保存文件并退出Vim。
    \end{enumerate}
  \item 在f2文件的第二行后插入如下内容:“With the development of society, the ability of reading becomes more and more important.”,并在最后一行的后面再添加一行,内容为:“We must know some methods to read faster.”。
    \begin{enumerate}
      \item 再次输入命令 \verb|vim f2|,启动Vim文本编辑器并打开f2文件。
      \item 移动光标到“When I was a schoolboy…”所在行,按【O】键,进入文本编辑模式,屏幕底部出现“--INSERT--”字样,Vim直接在第二行下新起一行。
      \item 输入“With the development of society, the ability of reading becomes more and more important.”
      \item 将光标移动到最后一行的末尾,按【Enter】键,另起一行,输入“We must know some methods to read faster.”。
    \end{enumerate}
  \item 将文本中所有的eyeballs替换为eye-balls。
    \begin{enumerate}
      \item
	按【Esc】键后输入“:”,进入末行模式。因为当前f2文件中共有5行,所以输入\\ \verb|1,5s/eyeballs/eye-balls/g|,并按【Enter】键,将文件中所有的eyeballs替换为eye-balls。
      \item 进入末行模式,输入 \verb|wq|,保存对文件的修改,并且退出Vim。
    \end{enumerate}
  \item 把第二行移动到文件末尾,删除第一行和第二行,随后撤销删除,最后不保存修改。
    \begin{enumerate}
      \item 再次输入命令 \verb|vim f2|,启动Vim文本编辑器并打开f2。
      \item 按【:】键,再次进入末行模式,输入 \verb|2m5|,将第二行移动到第五行的后面。
      \item 按【:】键,输入 \verb|1,2d|,删除第一行和第二行。
      \item 按【u】键,恢复被删除的部分。
      \item 按【:】键,进入末行模式,输入 \verb|q!|,退出Vim,不保存对文件的修改。
    \end{enumerate}
  \item 复制第二行,并添加到文件的末尾,然后删除第二行,保存修改后退出Vim。
    \begin{enumerate}
      \item 再次输入命令 \verb|vim f2|,启动Vim文本编辑器并打开f2文件。
      \item 按【:】键,进入末行模式,输入 \verb|2co5|,将第二行的内容复制到第五行的后面。
      \item 移动光标到第二行,按下两次【d】键,删除第二行。
      \item 按【:】键,输入 \verb|wq|,存盘并退出Vim。
    \end{enumerate}
  \item 新建userlist文件,要求从 /etc/passwd文件中取出用户名,并在文件头添加注释信息“This is a userlist generated by /etc/passwd.”,注释信息的前后各空一行,并添加“\#”符号设置这三行为注释信息。
    \begin{enumerate}
      \item 输入命令 \verb|vim userlist|,启动Vim文本编辑器并新建userlist文件。
      \item 按【:】键,进入末行模式,输入 \verb|r /etc/passwd|,在光标所在处读入 /etc/passwd文件的内容。
      \item 按【:】键,进入末行模式,输入 \verb|%s/:.*//g|,其中 \verb|%| 表示整个文档,而 \verb|:.*| 表示以 \verb|:| 开始的部分。最末一行显示进行了多少替换。\textbf{知识点解析:}在Vim中进行字符串替换操作时,可用 \verb|%| 表示整个文档;\verb|^| 表示行首。
      \item 按【i】键,切换到文本编辑模式,并移动光标至文件的第一行,输入注释信息“This is a userlist generated by /etc/passwd.”,并按【Enter】键添加两行空行。
      \item 按【Esc】键后输入 \verb|:1,3s/^/#/g|,其中 \verb|^| 表示行首。
      \item 最后按【:】键,输入 \verb|x|,存盘并退出Vim。
    \end{enumerate}
\end{enumerate}

\vspace{0.1in}
(三)强化练习Vim的常用操作
\begin{enumerate}
  \item 自由练习Vim的常用操作(如:启动,保存,退出,移动定位,修改删除,复制粘贴,搜索替换等)。
  \item 在线交互式学习Vim的基本操作:\href{http://www.openvim.com/tutorial.html}{http://www.openvim.com/tutorial.html}。
  \item 在线练习Vim的常用命令:\href{http://www.vimgenius.com/}{http://www.vimgenius.com/}。
  \item 观看Vim操作视频:\href{http://derekwyatt.org/}{http://derekwyatt.org/}。
  \item 在游戏中学习Vim:\href{http://vim-adventures.com/}{http://vim-adventures.com/}。
\end{enumerate}

