\chapter{Linux命令行界面的基本操作}

{\itshape
图形化用户界面下用户操作非常简单而直观,但到目前为止图形化用户界面还不能完成所有的操作任务。字符界面占用资源少,启动迅速,对于有经验的管理员而言,字符界面下使用shell命令更为直接高效。

shell命令是Linux操作系统的灵魂,灵活运用shell命令可完成操作系统所有的工作。并且,类Unix的操作系统在shell命令方面具有高度的相似性。熟练掌握shell命令,不仅有助于掌握Ubuntu/CentOS等单个的Linux发行版,而且几乎有助于掌握各种发行版本的Linux,甚至Unix。

包括Ubuntu/CentOS在内的Linux系统都具有虚拟终端。虚拟终端为用户提供多个互不干扰、独立工作的工作界面,并且在不同的工作界面可用不同的用户身份登录。也就是说,虽然用户只面对一个显示器,但可以切换到多个虚拟终端,好像在使用多个显示器。

Ubuntu/CentOS中不仅可在字符界面下使用shell命令,还可借助于桌面环境下的终端工具使用shell命令。桌面环境下的终端工具中使用shell命令可显示中文,而字符界面下默认仅显示英文。
}

\vspace{0.2in}
\noindent
一、实验要求
\begin{enumerate}
  \item 掌握图形化用户界面和字符界面下使用shell命令的方法。
  \item 掌握ls、cd等shell命令的功能。
  \item 掌握软链接和硬链接的使用与区别。
  \item 了解可读、可写和可执行权限对文件和目录的区别。
\end{enumerate}

\vspace{0.2in}
\noindent
二、实验准备
\begin{enumerate}
  \item 安装有Ubuntu/CentOS的计算机。
\end{enumerate}

\vspace{0.2in}
\noindent
三、实验内容

\vspace{0.1in}
(一)图形化用户界面下的shell命令操作

启动计算机,登录图形化用户界面。依次选择“应用程序”$\rightarrow$“系统工具”$\rightarrow$“终端”命令,打开桌面环境下的终端工具。
\begin{enumerate}
  \item 显示系统时间。
    \begin{enumerate}
      \item 输入命令 \verb|date|,显示系统的当前日期和时间。
    \end{enumerate}
  \item 查看日历。
    \begin{enumerate}
      \item 输入命令 \verb|cal 2018|,屏幕上显示出2018年的日历。
    \end{enumerate}
  \item 查看ls命令的-s选项的帮助信息。
    \begin{itemize}
      \item 方法一:
	\begin{enumerate}
	  \item 输入命令 \verb|man ls|,屏幕显示出手册页中ls命令相关帮助信息的第一页,介绍ls命令的含义、语法结构以及 \verb|-a|、\verb|-A|、\verb|-b| 和 \verb|-B| 等选项的含义。
	  \item 使用【PgDn】键、【PgUp】键以及上、下方向键找到 \verb|-s| 选项的说明信息。
	  \item 由此可知,ls命令的 \verb|-s| 选项等同于 \verb|--size| 选项,以文件块为单位显示文件和目录的大小。
	  \item 在屏幕上的“:”后输入q,退出ls命令的手册页帮助信息。
	\end{enumerate}
      \item 方法二:
	\begin{enumerate}
	  \item 输入命令 \verb|ls --help|,屏幕显示ls命令的中文帮助信息。
	  \item 拖动滚动条,找到-s选项的说明信息。
	  \item 在屏幕上的“:”后输入q,退出ls命令的手册页帮助信息。
	\end{enumerate}
    \end{itemize}
  \item 查看/etc目录下所有文件和子目录的详细信息。
    \begin{enumerate}
      \item 输入命令 \verb|cd /etc|,切换到 /etc目录。
      \item 输入命令 \verb|ls -al|,显示 /etc目录下所有文件和子目录的详细信息。
    \end{enumerate}
\end{enumerate}

\vspace{0.1in}
(二)字符界面下的shell命令操作

启动计算机后,默认进入Linux的图形化用户界面,此时按【Ctrl+Alt+F2】组合键切换到第一(CentOS)/二(Ubuntu)个虚拟终端(也可使用图形化界面中的终端)。输入一个普通用户的用户名和密码,登录系统。在字符界面下输入密码时,屏幕上不会出现类似“*”的提示信息,提高了密码的安全性。
\begin{enumerate}
  \item 查看当前目录。
    \begin{enumerate}
      \item 输入命令 \verb|pwd|,显示当前目录。
    \end{enumerate}
  \item 操作文本文件。
    \begin{enumerate}
      \item 创建文本文件。 

首先,用cat命令在用户主目录下创建一个名为f1的文本文件,内容为:
\begin{verbatim}
Linux is useful for us all.
You can never imagine how great it is.
\end{verbatim}
具体操作步骤如下:
    \begin{enumerate}
      \item 输入命令 \verb|cat >f1|,屏幕上输入点光标闪烁,依次输入上述内容。使用cat命令进行输入时,不能使用上、下、左、右方向键,只能用【Backspace】键来删除光标前一位置的字符。并且一旦按【Enter】键,该行输入的字符就不可修改了。
      \item 上述内容输入后,按【Enter】键,让光标处于输入内容的下一行,按【Ctrl+D】组合键结束输入。
      \item 要查看文件是否生成,输入命令 \verb|ls| 即可。
      \item 输入命令 \verb|cat f1|,查看f1文件的内容。
    \end{enumerate}
  \item 添加内容。

然后,向f1文件增加以下内容:Why not have a try?具体操作步骤如下:
    \begin{enumerate}
      \item 输入命令 \verb|cat >>f1|,屏幕上输入点光标闪烁。
      \item 输入上述内容后,按【Enter】键,让光标处于输入内容的下一行,按【Ctrl+D】组合键结束输入。
      \item 输入命令 \verb|cat f1|,查看f1文件的内容,会发现f1文件增加了一行。
    \end{enumerate}
    \textbf{知识点解析:}
    shell命令中可使用重定向来改变命令的执行。此处使用“\verb|>>|”符号可向文件结尾处追加内容,而如果使用“\verb|>|”符号则将覆盖已有的内容。shell命令中常用的重定向符号共3个,如下所示:
    \begin{itemize}
      \item \verb|>|:输出重定向,将前一命令执行的结果保存到某个文件。如果这个文件不存在,则创建此文件;如果这个文件已存在,则将覆盖原有内容。
      \item \verb|>>|:附加输出重定向,将前一命令执行的结果追加到某个文件。
      \item \verb|<|:将某个文件交由命令处理。
    \end{itemize}
  \item 统计文本文件信息。

    之后,统计f1文件的行数、单词数和字符数,并将统计结果存放到countf1文件中。具体操作步骤如下:
    \begin{enumerate}
      \item 输入命令 \verb|wc <f1 >countf1|,屏幕上不显示任何信息。
      \item 输入命令 \verb|cat countf1|,查看countf1文件的内容,其内容是f1文件的行数、单词数和字符数信息,即f1文件共有3行、19个单词和87个字符。
    \end{enumerate}
  \item 合并文件。

    最后,将f1和countf1文件合并为f文件。具体操作步骤如下:
    \begin{enumerate}
      \item 输入命令 \verb|cat f1 countf1 >f|,将两个文件合并为一个文件。
      \item 输入命令 \verb|cat f|,查看f文件的内容。
    \end{enumerate}
    \end{enumerate}
  \item 查看目录。
    \begin{enumerate}
      \item 分页显示 /etc目录中所有文件和子目录的信息。
    \begin{enumerate}
      \item 输入命令 \verb=ls /etc | more=,屏幕显示出 \verb|ls /etc| 命令输出结果的第一页,屏幕的最后一行还出现“--More--”或“--更多--”字样,按【Space】键可查看下一页信息,按【Enter】键可查看下一行信息。
      \item 浏览过程中按【Q】键,可结束分页显示。
    \end{enumerate}
    \textbf{知识点解析:}管道符号“\verb=|=”用于连接多个命令,前一命令的输出结果是后一命令的输入内容。
  \item 仅显示 /etc目录中的前5个文件和子目录。
    \begin{enumerate}
      \item 输入命令 \verb=ls /etc | head -n 5=,屏幕显示出 \verb|ls /etc|命令输出结果的前面5行。
    \end{enumerate}
    \end{enumerate}
  \item 清除屏幕内容。
    \begin{enumerate}
      \item 输入命令 \verb|clear|,或者使用快捷键【Ctrl+L】,屏幕内容将完全被清除,命令提示符定位在屏幕左上角。
    \end{enumerate}
\end{enumerate}

\vspace{0.1in}
(三)链接操作
\begin{enumerate}
  \item 创建原文件。
    \begin{enumerate}
      \item 使用 \verb|cd ~| 命令定位到主目录。
      \item 使用 \verb|touch original_file| 命令创建一个名为original\_file的文件。
      \item 运行 \verb|ls -l| 命令来查看刚才创建的文件。注意该文件的链接数目。
    \end{enumerate}
  \item 创建硬链接。
    \begin{enumerate}
      \item 使用 \verb|ln original_file hard_link| 命令对original\_file创建一个硬链接,该链接命名为hard\_link。
      \item 再次运行 \verb|ls -l|。注意文件的链接数目和上一次的修改时间。
      \item 运行 \verb|ls -i| 命令来显示文件的inode值。将会看到这两个文件的inode值是一样的。
    \end{enumerate}
  \item 创建软链接。
    \begin{enumerate}
      \item 使用 \verb|ln -s original_file soft_link| 命令为original\_file创建一个软链接,该链接命名为soft\_link。
      \item 再次使用 \verb|ls -l| 命令显示文件。注意文件的链接数目和修改时间。
      \item 使用 \verb|ls -i| 命令来查看所有文件的inode值。可以看出soft\_link的inode与original\_file和hard\_link的inode不同。
    \end{enumerate}
  \item 原文件对链接文件的影响。
    \begin{enumerate}
      \item 使用 \verb|cat| 命令来查看每个文件的内容,确认这些文件中没有包含任何文本或数据。
      \item 要了解原文件的改变是如何影响链接文件的,可以使用\\ \verb|echo "This text goes to the original_file" >>original_file| \\ 命令给original\_file中添加一行“This text goes to the original\_file”。
      \item 运行 \verb|ls -l| 命令来查看原文件和链接文件在文件大小方面的变化。虽然我们只给original\_file文件添加了数据,但hard\_link文件的大小也会发生变化,而soft\_link文件的大小没有变化。
      \item 使用 \verb|cat| 命令来查看文件的内容,可以发现三个文件具有完全相同的内容。
    \end{enumerate}
  \item 硬链接对原文件的影响。
    \begin{enumerate}
      \item 要了解硬链接文件的变化对原始文件的影响,可以使用\\ \verb|echo "This text goes to the hard_link file" >>hard_link| \\ 命令给hard\_link中添加一行文本“This text goes to the hrad\_link file”。
      \item 运行 \verb|ls -l| 并观察输出。original\_file和hard\_link的修改时间和大小都发生了改变,但是soft\_link没有变化。
      \item 现在再次使用 \verb|cat| 命令来显示文件的内容,注意每个文件的变化。
    \end{enumerate}
  \item 软链接对原文件的影响。
    \begin{enumerate}
      \item 如果使用 \verb|echo| 命令给soft\_link文件添加一行文本,将会修改原始文件,从而也更新了hard\_link文件。使用\\ \verb|echo "This text goes to the soft_link file" >>soft_link| \\ 命令给soft\_link文件中添加一行“This text goes to the soft\_link file”。
      \item 使用 \verb|cat| 命令来显示文件的内容。
    \end{enumerate}
\end{enumerate}

\vspace{0.1in}
(四)链接的本质
\begin{enumerate}
  \item 创建原文件、硬链接和软链接。
    \begin{enumerate}
      \item 使用 \verb|cd ~| 命令切换到主目录。
      \item 分别使用 \verb|touch fn|、\verb|ln fn hl| 和 \verb|ln -s fn sl| 命令创建名为fn的原文件、名为hl的硬链接、名为sl的软链接。
      \item 使用 \verb|ls -li fn hl sl| 查看各个文件的属性。(注意:fn、hl、sl的inode、链接数、文件大小;sl的指向【除了ls的输出外,还可以使用 \verb|readlink sl| 查看sl的指向】)。
      \item 使用 \verb|echo 123 > fn| 向原文件添加内容。
      \item 使用 \verb|ls -li fn hl sl| 查看各个文件的属性。(注意:fn、hl、sl的inode、链接数、文件大小的变化)。
      \item 使用 \verb|cat fn hl sl| 查看各个文件的内容。
    \end{enumerate}
  \item “不小心”删除原文件。
    \begin{enumerate}
      \item 使用 \verb|rm fn| 删除原文件
      \item 使用 \verb|ls -li hl sl| 查看各个文件的属性。(注意:hl、sl的inode、链接数、文件大小的变化;sl指向的有效性)。
      \item 使用 \verb|cat hl sl| 查看各个文件的内容。
    \end{enumerate}
  \item “碰巧”创建同名文件
    \begin{enumerate}
      \item 使用 \verb|touch fn| 创建一个和被删除文件同名的文件。
      \item 使用 \verb|ls -li fn hl sl| 查看各个文件的属性。(注意:fn、hl、sl的inode、链接数、文件大小的变化;sl指向的有效性)。
      \item 使用 \verb|cat fn|、\verb|cat hl| 和 \verb|cat sl| 分别查看各个文件的内容。
      \item 使用 \verb|echo abcdef > fn| 向新的文件添加内容。
      \item 使用 \verb|ls -li fn hl sl| 查看各个文件的属性。(注意:fn、hl、sl的inode、链接数、文件大小的变化;sl指向的有效性)。
      \item 使用 \verb|cat fn hl sl| 查看各个文件的内容。
    \end{enumerate}
\end{enumerate}

\textbf{知识点解析:}硬链接指向的是原文件所在的数据块,除了名字不同以外,本质上和原文件没有区别。软链接存储的仅仅是原文件的路径,并非实际的数据块。

\vspace{0.1in}
(五)目录的可读、可写、可执行权限
\begin{enumerate}
  \item 创建目录。
    \begin{enumerate}
      \item 使用 \verb|cd ~| 命令切换到主目录。
      \item 使用 \verb|mkdir dir| 创建目录,使用 \verb|touch dir/file1| 在目录中创建文件。
      \item 使用 \verb|ls -ld dir| 查看所有者对目录的权限。
      \item 分别尝试 \verb|ls dir|、\verb|touch dir/file2| 和 \verb|cd dir; cd ..|(注意命令的提示或者最终状态)。
    \end{enumerate}
  \item 目录的可读权限。
    \begin{enumerate}
      \item 使用 \verb|chmod u-r dir| 删除所有者对目录的可读权限。
      \item 使用 \verb|ls -ld dir| 查看所有者对目录的权限。
      \item 分别尝试 \verb|ls dir|、\verb|touch dir/file3| 和 \verb|cd dir; cd ..|(注意命令的提示或者最终状态)。
      \item 使用 \verb|chmod u=rwx dir| 恢复所有者对目录的原有权限。
      \item 使用 \verb|ls -ld dir| 查看所有者对目录的权限。
    \end{enumerate}
  \item 目录的可写权限。
    \begin{enumerate}
      \item 使用 \verb|chmod u-w dir| 删除所有者对目录的可读权限。
      \item 使用 \verb|ls -ld dir| 查看所有者对目录的权限。
      \item 分别尝试 \verb|ls dir|、\verb|touch dir/file4| 和 \verb|cd dir; cd ..|(注意命令的提示或者最终状态)。
      \item 使用 \verb|chmod u=rwx dir| 恢复所有者对目录的原有权限。
      \item 使用 \verb|ls -ld dir| 查看所有者对目录的权限。
    \end{enumerate}
  \item 目录的可执行权限。
    \begin{enumerate}
      \item 使用 \verb|chmod u-x dir| 删除所有者对目录的可读权限。
      \item 使用 \verb|ls -ld dir| 查看所有者对目录的权限。
      \item 分别尝试 \verb|ls dir|、\verb|touch dir/file5| 和 \verb|cd dir|(注意命令的提示或者最终状态)。
      \item 使用 \verb|chmod u=rwx dir| 恢复所有者对目录的原有权限。
      \item 使用 \verb|ls -ld dir| 查看所有者对目录的权限。
    \end{enumerate}
\end{enumerate}

\textbf{知识点解析:}目录的可读权限意味着可以使用ls列出目录的内容(就像文件的可读权限意味着可以读取文件的内容一样)。目录的可写权限意味着可以在目录中使用touch创建文件(也就是修改目录的内容,就像文件的可写权限意味着可以修改文件的内容的一样)。目录的可执行权限意味着可以使用cd切换到目录中去(同时也会对ls和touch等命令有所影响)。所以,目录一定会有可执行权限,绝大多数情况下也会同时有可读权限,换言之,目录应该至少同时有可读和可执行权限。


\vspace{0.1in}
(六)常见文件与文件夹操作的命令实现

回忆、总结图形界面下文件与文件夹的常见操作(如:新建、删除、重命名、移动等),在命令行界面中用命令将其实现。
